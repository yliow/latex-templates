subs: \answerbox{
c: (\,92) c: (a,97) c: (n,110) c: (s,115) c: (w,119) c: (e,101) c: (r,114) c: (b,98) c: (o,111) c: (x,120) c: ({,123) 
subs: \begin{answercode}

c: (\,92) c: (b,98) c: (e,101) c: (g,103) c: (i,105) c: (n,110) c: ({,123) c: (a,97) c: (n,110) c: (s,115) c: (w,119) c: (e,101) c: (r,114) c: (c,99) c: (o,111) c: (d,100) c: (e,101) c: (},125) c: (
,10) 
subs: \begin{answerlong}

c: (\,92) c: (b,98) c: (e,101) c: (g,103) c: (i,105) c: (n,110) c: ({,123) c: (a,97) c: (n,110) c: (s,115) c: (w,119) c: (e,101) c: (r,114) c: (l,108) c: (o,111) c: (n,110) c: (g,103) c: (},125) c: (
,10) 
xs: ['1 2 3\n4 5 6', '\nint x;\n', '55555']
number indices: 3
[(228, 239), (419, 428), (578, 584)]
number answers: 3
['aaa', 'bbb', 'ccc']
\input{thispreamble.tex}

\renewcommand\AUTHOR{jdoe5@cougars.ccis.edu} % CHANGE TO YOURS

\begin{document}
\topmattertwo

%------------------------------------------------------------------------------
\nextq
1 + 1 = \answerbox{aaa}

%------------------------------------------------------------------------------
\nextq
Write a C++ statement to declare integer variable \verb!x!.
\\
\ANSWER
\begin{answercode}
bbb
\end{answercode}

%------------------------------------------------------------------------------
\nextq
What is 1 + 1?
\\
\ANSWER
\begin{answerlong}
ccc
\end{answerlong}

%------------------------------------------------------------------------------
\newpage
\input{instructions.tex}
\end{document}

